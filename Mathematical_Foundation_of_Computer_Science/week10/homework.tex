\documentclass[12pt]{article}
\usepackage{amsmath}
\usepackage{amssymb}
\usepackage{amsthm}
\usepackage{enumerate}
\usepackage{hyperref}
\usepackage{xypic}
\usepackage{txfonts}
\usepackage{amsmath}
\usepackage{amssymb}
\usepackage{amscd}
\usepackage{amsmath, mathtools,amssymb}
\usepackage{amsfonts,semantic,colortbl,mathrsfs,stmaryrd}
\usepackage{enumerate}
\usepackage{multirow}
\usepackage{graphicx}
\date{Feb 14, 2012}
\newtheorem{thm}{Theorem}
\newtheorem{lemma}[thm]{Lemma}
\newtheorem{fact}[thm]{Fact}
\newtheorem{cor}[thm]{Corollary}
\newtheorem{eg}{Example}
\newtheorem{hw}{Problem}
\newcommand{\xor}{\otimes}
\newenvironment{sol}
  {\par\vspace{3mm}\noindent{\it Solution}.}
  {\qed}
\begin{document}
\begin{center}
{\LARGE\bf Homework 8}\\
\vspace{2mm}
\footnotesize{$*$ Name:\underline{Xin Xu}  \quad Student ID:\underline{519021910726} \quad Email: \underline{xuxin20010203@sjtu.edu.cn}}
\vspace{2mm}
\end{center}

\begin{hw}
  Which of the following statements about graph $G$ and $H$ are true?
  \begin{enumerate}
    \item $G$ and $H$ are isomorphic if and only if for every map $f:V(G)\rightarrow V(H)$ and for any two vertices $u,v\in V(G)$, we have $\{u,v\}\in E(G)\Leftrightarrow \{f(u),f(v)\}\in E(H)$.
    \item $G$ and $H$ are isomorphic if and only if there exists a bijection $f: E(G)\rightarrow E(H)$.
    \item If there exists a bijection $f:V(G)\rightarrow V(H)$ such that every vertex $u\in V(G)$ has the same degree as $f(u)$, then $G$ and $H$ are isomorphic.
    \item If $G$ and $H$ are isomorphic, then there exists a bijection $f:V(G)\rightarrow V(H)$ such that every vertex $u\in V(G) $ has the same degree as $f(u)$.
    \item If $G$ and $H$ are isomorphic, then there exists a bijection $f: E(G)\rightarrow E(H)$.
    \item $G$ and $H$ are isomorphic if and only if there exists a map $f:V(G)\rightarrow V(H)$ such that for any two vertices $u,v\in V(G)$, we have $\{u,v\}\in E(G)\Leftrightarrow \{f(u),f(v)\}\in E(H)$.
    \item Every graph on $n$ vertices is isomorphic to some graph on the vertex set $\{1,2,\ldots, n\}$.
    \item Every graph on $n\geq 1$ vertices is isomorphic to infinitely many graphs.
  \end{enumerate}
\end{hw}

\begin{sol}
    Statement 4,5,6 and 7 are true.
\end{sol}

\begin{hw}% \textcolor{blue}{(This exercise is meant to correct a false statement in the class.)}

\noindent Two simple graphs $G=(V,E)$ and $G'=(V',E')$. A map $f: V\rightarrow V'$. Now if $f$ satisfies:
\begin{enumerate}[i)]
  \item It is a bijective function;
  \item $\{x,y\}\in E$ if and only if $\{f(x), f(y)\}\in E'$;
\end{enumerate}
Then we say that graph $G$ and $G'$ are \emph{isomorphic} to each other. We use  $G\cong G'$ to stand for the isomorphism relation.

Consider the following questions:
\begin{enumerate}
  \item $G=K_n$ (Recall: $K_n$ is a clique with $n$ vertices), $g: V\rightarrow V'$ is a function which only satisfies requirement ii). Prove that $G'$ must contain a subgraph which is a clique with $n$-vertices.
  \item $G=K_{n,m}$ (Recall: $K_{n,m}$ is the so-called \emph{complete bipartite graphs}), $g$ is the same as in question 1.  What will be the simplest $G'$ that is related to $G$ under the new relation.
\end{enumerate}
\end{hw}

\begin{sol}
    \begin{enumerate}
        \item \begin{proof}
        Since $G$ has $n$ vertices and function $g$ is a map from $V$ to $V'$, $Ran(g)\leq n$. Since $G'$ contains $n$ vertices, $g$ is a bijective function. So, $G$ and $G'$ are isomorphic, which means the statement is true.
        \end{proof}
        \item The simplest graph $G'$ is a graph contains two vertices and an edge that connects the two vertices.
    \end{enumerate}
\end{sol}


\begin{hw}
How many graphs on the vertex set $\{1,2,\ldots,2n\}$ are isomorphic to the graph consisting of $n$ vertex-disjoint edges (i.e. with edge set \{\{1,2\},\{3,4\},\ldots, \{2n-1,2n\}\}?
\end{hw}

\begin{sol}
    Suppose the number of graphs isomorphic to graph with edge set \{\{1,2\},\{3,4\},\ldots, \{2n-1,2n\}\} is $T(2n)$. By induction, $T(2n)=(2n-1)T(2n-2)=(2n-1)(2n-3)T(2n-4)=\ldots=(2n-1)!!$.
\end{sol}

\begin{hw}
Construct an example of a sequence of length $n$ in which each term is some of the numbers $1,2,\ldots, n-1$ and which has an even number of odd terms, and yet the sequence is not a graph score. Show why it is not a graph score.
\end{hw}

\begin{sol}
    We can know that $n>3$. We reorder the sequence so that the value is increasing. The first element is 1 and the last two elements are $n-1,n-1$. The other number can be randomly chosen to satisfies the condition that "has an even number of odd terms". \\
    To prove it not a graph score, we first delete the last element $n-1$ and change the value of remainded number. So, the first element becomes 0 and the last element becomes $n-2$. When we delete $n-2$, the action cannot continue, so the sequence isn't a graph score.
\end{sol}


\begin{hw}
Let $G$ be a graph with 9 vertices, each of degree 5 or 6. Prove that it has at least 5 vertices of degree 6 or at least 6 vertices of degree 5.
\end{hw}

\begin{proof}
    We will prove it by contradiction. Hypothesis that $G$ at most has 4 vertices of degree 6 and at most has 5 vertices of degree 5. Since $G$ has 9 vertices and the degree of each vertex is 5 or 6, $G$ has exactly 4 vertices of 6 and 5 vertices of 5.\\
    The total degree is $4\times 6+5\times 5=49$. According to handshake lemma, the answer is wrong. So the hypothesis is wrong. The original statement is true.
\end{proof}

\begin{hw}
Given a sequence $(d_1, d_2, \ldots, d_n)$ of  positive integers (where $n\geq 1$):
\begin{enumerate}[(i)]
  \item There exists a tree with score $(d_1, d_2, \ldots, d_n)$.
  \item $\sum_{i=1}^{n}d_i=2n-2$.
\end{enumerate}
Prove that (i) and (ii) are equivalent.
\end{hw}

\begin{proof}
    $(i)\leftarrow (ii)$:\\
    The number of edges of a tree with $n$ vertices is $n-1$. According to handshake lemma, $\sum_{i=1}^{n}d_i=2(n-1)=2n-2$. The statement is proved.\\
    $(i)\leftarrow (ii)$:\\
    According to handshake lemma, the number of edges is $n-1$. For there are $n$ vertices with the sum of all degree is $2n-2$, there must be a vertex with degree $1$. Regard it as $v_n$. It's a leaf obviously.
    We remove $v_n$ from the score, and there remains $n-1$ vertices and the sum of degree becomes $2n-2-2=2(n-1)-2$. So, by induction, everytime we remove one leaf from the graph, and the graph finally becomes empty, which satisfies the definition of the tree. The statement is proved.
\end{proof}


\begin{hw}
  Let $N_k$ denote the number of spanning trees of $K_n$ in which the vertex $n$  has degree $k$, $k=1,2,\ldots, n-1$ (recall that we assume $V(K_n)=\{1,2,\ldots,n\}$).
  \begin{enumerate}[i)]
    \item Prove that $(n-1-k)N_k= k(n-1)N_{k+1}$.
    \item Using i), derive $N_k= {n-2 \choose k-1}(n-1)^{n-1-k}$.
    \item Prove Cayley's formula from ii).
  \end{enumerate}
\end{hw}

\begin{sol}
    \begin{enumerate}
        \item From $N_k$ to $N_{k+1}$, we use this method to construct a new spanning tree: add a new node to $v_n$ and delete the edge that the new part attached to originally. So, $(n-1-k)N_k$ is the ways to construct $N_{k+1}$. But there are some repetition: for the tree $N_{k+1}$, every node $v_i$ except $v_n$ can attach to the other $k$ nodes $v_n$ directly attaches to except the node in $v_i$'s branch. So, there are $k(n-1)$ repetitions. 
        And in conclusion, $N_{k+1}=\frac{(n-1-k)}{k(n-1)}N_k $. So, the statement is true.
        \item $N_k=\frac{(n-k)}{(k-1)(n-1)}N_{k-1}=\frac{(n-k)(n-k+1)}{(k-1)(k-2)(n-1)^2}N_{k-2}=\ldots =\frac{(n-k)(n-k+1)\ldots (n-2)}{(k-1)(k-2)\ldots 1(n-1)^{k-1}}N_1=\binom{n-2}{k-1}\frac{N_1}{(n-1)^{k-1}}  $. Since $N_{n-1}=1$, we know from the equation above that $N_{n-1}=\frac{N_1}{(n-1)^{n-2}}=1 $. So, $N_1=(n-1)^{n-2}$.
        As a result, $N_k=\binom{n-2}{k-1}\frac{N_1}{(n-1)^{k-1}}=\binom{n-2}{k-1}(n-1)^{n-1-k} $.
        \item We assume $C(n)$ as the number of all different spanning trees with $n$ nodes. $C(n)=N_1+N_2+\ldots +N{n-1}=\binom{n-2}{0}(n-1)^{n-2}+ \binom{n-2}{1}(n-1)^{n-3}+\ldots \binom{n-2}{n-2}(n-1)^0=(n-1+1)^{n-2}=n^{n-2}$. So, Cayley's formula is proved.
    \end{enumerate}
\end{sol}

\end{document}

%%% Local Variables:
%%% mode: tex-pdf
%%% TeX-master: t
%%% End: