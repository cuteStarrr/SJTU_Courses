\documentclass[12pt]{article}
\usepackage{amsmath}
\usepackage{amssymb}
\usepackage{amsthm}
\usepackage{enumerate}
\usepackage{hyperref}
\usepackage{xypic}
\usepackage{txfonts}
\usepackage{amsmath}
\usepackage{amssymb}
\usepackage{amscd}
\usepackage{amsmath, mathtools,amssymb}
\usepackage{amsfonts,semantic,colortbl,mathrsfs,stmaryrd}
\usepackage{enumerate}
\usepackage{multirow}
\usepackage{graphicx}
\date{Feb 14, 2012}
\newtheorem{thm}{Theorem}
\newtheorem{lemma}[thm]{Lemma}
\newtheorem{fact}[thm]{Fact}
\newtheorem{cor}[thm]{Corollary}
\newtheorem{eg}{Example}
\newtheorem{hw}{Problem}
\newcommand{\xor}{\otimes}
\newenvironment{sol}
  {\par\vspace{3mm}\noindent{\it Solution}.}
  {\qed}
\begin{document}
\begin{center}
{\LARGE\bf Homework 9}\\
\vspace{2mm}
\footnotesize{$*$ Name:\underline{Xin Xu}  \quad Student ID:\underline{519021910726} \quad Email: \underline{xuxin20010203@sjtu.edu.cn}}
\vspace{2mm}
\end{center}

\begin{hw}
Find an example to verify the claim that `(pairwise) independence does not  imply mutual independence'.  Pls give a detailed proof.
\end{hw}

\begin{sol}
    Suppose there are three events: $A,B,C$ so that $Pr(A)=Pr(B)=Pr(C)=\frac{1}{5}, Pr(AB)=Pr(BC)=Pr(AC)=\frac{1}{25}, Pr(A\cup B\cup C)=\frac{13}{25}$.
    It's pairwise independent but not mutually independent. Proof is as follows:\\
    Since $Pr(AB)=\frac{1}{25}=\frac{1}{5}\times \frac{1}{5}=Pr(A)\times Pr(B)$, and it's also true for both $Pr(AC)$ and $Pr(BC)$. So, it's clear that events $A,B,C$ are pairwise independent. But for $Pr(ABC)$, we can calculate that:\\
    $Pr(A\cup B\cup C)=Pr(A)+Pr(B)+Pr(C)-Pr(AB)-Pr(AC)-Pr(BC)+Pr(ABC)$, so that $Pr(ABC)= \frac{1}{25} \neq Pr(A)\times Pr(B)\times Pr(C)$.\\
    Above all, this example denotes that `(pairwise) independence does not imply mutual independence'.
\end{sol}


\begin{hw}
Show that, if $E_1, E_2, \ldots, E_n$ are mutually independent, then so are $\overline{E_1}, \overline{E_2},\ldots, \overline{E_n}$.
\end{hw}

\begin{proof}
    We will prove it by induction.
    For any $k$ events $E_{i_1},E_{i_2},\ldots ,E_{i_k}$, we can know that:
    \begin{align*}
    Pr(E_{i_1}E_{i_2}\ldots \overline{E_{i_k}})= & Pr(E_{i_1}E_{i_2}\ldots E_{i_{k-1}}) - Pr(E_{i_1}E_{i_2}\ldots E_{i_k})\\
    = & Pr(E_{i_1})Pr(E_{i_2})\ldots Pr(E_{i_{k-1}})-Pr(E_{i_1})Pr(E_{i_2})\ldots Pr(E_{i_{k}})\\
    = & Pr(E_{i_1})Pr(E_{i_2})\ldots (1-Pr(E_{i_{k}})).
    \end{align*}
    So, every time we change an event $E_{i_j}$ into $\overline{E_{i_j}}$, the corresponding probability changes to $1-Pr(E_{i_j})$. After $k$ times recursion, $Pr(\overline{E_{i_1}}\overline{E_{i_2}}\ldots \overline{E_{i_k}})=(1-E_{i_1})(1-E_{i_2})\ldots (1-E_{i_k})$.\\
    The statement is true.
\end{proof}


\begin{hw}
 A monkey types on a 26 -letter keyboard that has lowercase letters only.
Each letter is chosen independently and uniformly at random from the alphabet. If the
monkey types 1,000,000 letters. what is the expected number of times the sequence
``proof'' appears?
\end{hw}

\begin{sol}
    There are $1000000-4=999996$ ways to choose $5$ continuous chars from $1000000$ chars. And for every continuous chars, the probability of sequence "proof" appears is $Pr(proof)=\frac{1}{26^5}$. So that the expected numbe rof times is $999996\times Pr(proof)=\frac{999996}{26^5}$.
\end{sol}


\begin{hw}
We have 27 fair coins and one counterfeit coin (28 coins in all), which looks like a fair coin but is a bit heavier. Show that one needs at least 4 weighings to determine the counterfeit coin. We have no calibrated weights, and in one weighing we can only find out which of two groups of some $k$ coins each is heavier, assuming that if both groups consist of fair coins only the result is an equilibrium.
\end{hw}

\begin{sol}
    We will prove that $3$ times weighting cannot find out the counterfeit coin.\\
    Every weighting, we can divide all the coins into three parts and choose one part from them.Since $28>3^3$, we cannot find out the counterfeit coin in $3$ times weighting. 
    Consider the worst condition, the first choice can reduce the number to $10:(9,9,10)$. The second weighting can reduce the number to $4:(3,3,4)$. Since the number can only increase by adding the normal coins, there must be $2$ coins grouped into one:$(1,1,2)$. So that if the third weighting reduced the number to $2$, we cannot find out the counterfeit coin.
\end{sol}


\begin{hw}
\begin{enumerate}
  \item Prove that, for every integer $n$, there exists a coloring of the edges of the complete graph $K_n$ by two colors so that the total number of monochromatic copies of $K_4$ is at most ${n\choose 4}2^{-5}$.
  \item Give a randomized algorithm for finding a coloring with at most ${n \choose 4}2^{-5}$ monochromatic (i.e. single-color) copies of $K_4$ that runs in expected time polynomial in $n$.
\end{enumerate}
\end{hw}

\begin{proof}
    \begin{enumerate}
        \item For any $K_4$, the probability of being monochromatic is $\frac{1}{2^{\binom{4}{2} }}\times 2=\frac{1}{2^5}$. To choose $4$ nodes from $n$ nodes to form a $K_n$, there is at most $\binom{n}{4} $ ways to form a $K_n$. So that the number is at most ${n\choose 4}2^{-5}$.
        \item Every time the algorithm finds a $K_4$ out of $K_n$, and examine it monochromatic or not. The loop needs $\binom{n}{4} $ times and the examination needs $6$ times. So the total time is $6\binom{n}{4} $ satisfying polynomial in $n$.
    \end{enumerate}
\end{proof}


\begin{hw}
Use the Lovasz local lemma to show that if \[4 {k \choose 2} {n \choose {k-2}}2^{1-{k\choose 2}}\leq 1\]
then it is possible to color the edges of $K_n$ with two colors so that it has no monochromatic (i.e. single color) $K_k$ subgraph.
\end{hw}

\begin{proof}
    We assume that the event $E_i$ is: chosen $K_k^i$ is monochromatic. So that $Pr(E_i)=2^{1-\binom{k}{2} }$. And for every event $E_i$, if $E_i$ with other events $E_j$ can combine to form a monochromatic $K_k$, there is an edge between node $E_i$ and node $E_j$.
    So that $d< {k \choose 2} {n \choose {k-2}}$. With conditions above, $4dPr(E_i)\leq 1$, so that $Pr(\overline{E_1}\overline{E_2}\ldots \overline{E_n})\geq 0$, which means it is possible to color the edges of $K_n$ with two colors so that it has no monochromatic (i.e. single color) $K_k$ subgraph.
\end{proof}


\begin{hw}
What is the expected number of trees with $k$ vertices in $G\in \mathcal{G}(n,p)$?
\end{hw}

\begin{sol}
    For any $k$ nodes, the probability to form a tree is $Pr(T)= k^{k-2}p^{k-1}(1-p)^{\binom{k-2}{2} }$.
    And there are $\binom{n}{k} $ ways to find $k$ nodes. So that the expected number is $\binom{n}{k}k^{k-2}p^{k-1}(1-p)^{\binom{k-2}{2} } $.
\end{sol}


\begin{hw}
Show that if almost all $G\in \mathcal{G}(n,p)$ have a graph property $\mathcal{P}_1$ and almost all $G\in \mathcal{G}(n,p)$ have a graph property $\mathcal{P}_2$, then almost all $G\in \mathcal{G}(n,p)$ have both properties.
\end{hw}

\begin{proof}
    \begin{align*}
        \lim_{n \to \infty} Pr(\mathcal{P}_1\mathcal{P}_2)=&\lim_{n \to \infty}Pr(\mathcal{P}_1)+Pr(\mathcal{P}_2)-Pr(\mathcal{P}_1\cup \mathcal{P}_2)\\
        =&\lim_{n \to \infty} 1+1-Pr(\mathcal{P}_1\cup \mathcal{P}_2)\\
        \geq & \lim_{n \to \infty} 1+1-1\\
        \geq & 1
    \end{align*}
    So that almost all $G\in \mathcal{G}(n,p)$ have both properties.
\end{proof}



\end{document}

%%% Local Variables:
%%% mode: tex-pdf
%%% TeX-master: t
%%% End: